\chapter*{Introduzione}
\addcontentsline{toc}{chapter}{Introduzione}
In un mondo, sempre più urbanizzato, dove la percentuale di anziani cresce sempre di più, nasce il pensiero rispetto a quali debbano essere i canoni per una città del futuro. Da qui, nasce il concetto di \textit{\textbf{walkability}}. Generalmente, il termine walkability, indica il livello di comfort e sicurezza per i pedoni in una certa area urbana.
\\In concreto, una città ha un buon livello di walkability, se le strade e i marciapiedi sono in buono stato, se è presente una segnaletica stradale chiara e se sono presenti negozi, in una distanza facilmente percorribile a piedi.
\\Per questo, l'esigenza costante di individui che, vogliono rimanere sempre connessi al mondo che li circonda, condividendo sempre di più le proprie esperienze, ha portato alla creazione di un'applicazione mobile.
\\Il progetto Urban Stories Sharing, nasce da un'idea del dipartimento di Informatica, Sistemistica e Comunicazione dell'Università degli Studi di Milano-Bicocca, ed ha come obiettivo sia quello di migliorare la walkability di un'area urbana, ma soprattutto, quello di spingere le persone anziane ad essere ancora partecipi nella società, ad uscire nelle strade e quindi permettergli di raccogliere i dati relativi all'area urbana da loro frequentata, descrivendo storie attraverso: note scritte, foto, video e note vocali.
\\Per permettere tutto ciò, come enunciato qui sopra, le strade devono essere facilmente percorribili, in buono stato e senza pericoli.
\\Tutti questi dati geolocalizzati verrano salvati su un repository centralizzato, così che potranno essere condivise fra gli utenti.
\\La realizzazione del sistema è stata suddivisa in due parti: lo sviluppo del \textit{\textbf{front-end}} e lo sviluppo del \textit{\textbf{back-end}}.
\\In questo rapporto di stage, si illustrerà la parte di sviluppo del \textit{\textbf{front-end}}.
\\Nel capitolo \ref{chapter1} verranno enunciate le specifiche dell'applicazione, nel capitolo \ref{chapter2} verranno ampiamente discusse le scelte implementative.
\\Successivamente nel capitolo \ref{chapter3} verrà proposto un manuale utente, in cui si spiegherà come utilizzare l'applicazione, non dal punto di vista tecnico, ma come poter fare una determinata azione, e, si terminerà, con il capitolo \ref{chapter4}, in cui verranno presentate le conclusioni e i possibili sviluppi futuri.
