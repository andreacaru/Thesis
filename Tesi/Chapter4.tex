\chapter{Conclusioni e sviluppi futuri}
\label{chapter4}
In questo capitolo verrano esposti i possibili  sviluppi futuri del progetto e le conclusioni di questo rapporto di stage.

\section{Il futuro}
L'applicazione Urban Stories Sharing, attualmente, soddisfa tutti i requisiti di sistema prefissati all'inizio del percorso. Tuttavia ci sono alcuni aspetti che potrebbero essere migliorati ed implementati in una versione futura.
\\Principalemente sono:
\begin{enumerate}
    \item Miglioramento del design e User Experience
    \item Implementare cancellazione multipla degli elementi multimediali
    \item Miglioramento delle performance
    \item Splash Screen e logo app
    \item Tutorial iniziale
    \item Implementare sistema di autenticazione
\end{enumerate}
Per il primo punto potrebbe essere necessario contattare e prendere in considerazione un esperto di usabilità che permetta di implementare un'interfaccia mirata e ridisegnata, migliorarla in base al target scelto.
\\\\Per il secondo punto si potrebbe pensare ad un'interfaccia stile "galleria" in cui l'utente possa selezionare simultaneamente diversi file multimediali per poterli cancellarli contemporaneamente.
\\\\Per il terzo punto si potrebbero migliorare le prestazioni intervenendo sulla qualità del codice e suddividere in più classi i vari compiti definiti all'interno dell'applicazione.
\\\\Per il quarto punto si potrebbe contattare e prendere in cosiderazione un grafico o designer che sia disposto a disegnare un logo per l'applicazione per poi creare uno Splash Screen, ovvero una schermata iniziale di "caricamento", che dia un feedback di attesa piacevole all'utente intanto che l'applicazione carichi tutte le sue funzionalità, per non lasciarlo in "attesa" inutilmente.
\\\\Per il quinto punto si potrebbe implementare un tutorial iniziale, al primo avvio dell'applicazione, che permetta all'utente di capire che cosa può fare e come farlo.
\\\\Per l'ultimo punto si potrebbe implementare un sistema di autenticazione in modo tale che ogni utente possa accedere al sistema, vedere uno storico delle note inserite, per poi visionarle all'interno della mappa, così che gli utenti possano interagire fra di loro. In questo caso ovviamente ci sarebbero da fare delle modifiche al \textit{\textbf{back-end}}.

\section{Conclusioni}
Il lavoro svolto durante questo progetto di stage ha permesso di realizzare un'applicazione Android funzionante, in tutti i suoi aspetti, ovvero la visualizzazione della propria posizione geolocalizzata all'interno di una mappa fornita da Google Maps, l'utilizzo delle risorse hardware del dispositivo, come fotocamera, videocamera e microfono, la richiesta dei vari permessi per utilizzare le varie funzionalità, il salvataggio in locale dei vari file multimediali, l'invio di quest'ultimi (foto, video, note vocali e scritte) convertiti in \textit{\textbf{Base64}} ad un repository centralizzato attraverso una chiamata \textit{\textbf{HTTP}} di tipo \textit{\textbf{POST}}, senza tralasciare la possibilità di cancellarli dalla memoria fisica del dispositivo, ed infine, gestire i vari messaggi di avviso per l'utente come Snackbar, Alert Dialog e un \textit{\textbf{AsyncTask}} durante la chiamata POST.
\\\\Questo percorso mi ha permesso di ampliare e migliorare le mie conoscenze in merito ai linguaggi di programmazione come Java ed XML, conoscere il sistema IDE Android Studio, che mi era sconosciuto all'inizio, imparare a gestire un progetto di questa portata in autonomia, organizzando i vari step decisionali e di implementazione, risolvendo problemi ricorrenti trovando soluzioni flessibili e coerenti con quello che era il progetto. 
\\Tutto questo sotto la supervisione del mio tutor Davide Ginelli. 
\\Approfittandone, colgo l'occasione per ringraziarlo per l'aiuto ed il supporto puntuale e professionale ricevuto durante tutto il percorso.